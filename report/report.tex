\documentclass{article}
\usepackage[utf8]{inputenc}

\title{High Availability Solutions for HPC Clusters with Kubernetes}
\author{ Badr-Eddine Aji}
\date{}

\begin{document}

\maketitle

\section{Introduction}
\begin{itemize}
    \item What is HPC ?
     
\begin{itemize}
    \item  Define the concept of High-Performance Computing (HPC) and its significance in scientific research and technological innovation

\end{itemize}
    \item HPC Clusters: Management vs. Computing Nodes
\begin{itemize}
    \item   Discuss the distinction between managing HPC clusters and computing nodes, highlighting the different roles and requirements of these components.

\end{itemize}
    
    \item CISM :
        \begin{itemize}

            \item Presentation of the typical (bare-metal) architecture of an HPC cluster
            \item LEMAITRE and MANNBACK
            \item Management and Computing jobs (How many jobs per day?)

        \end{itemize}
    

    \item The Role of Slurm in HPC Clusters:
            \begin{itemize}
            \item examine the crucial role of the Slurm job scheduler in managing jobs and resources within HPC clusters, emphasizing its importance for the smooth operation of these systems
        \end{itemize}
     \item Problem statement and objectives
        \begin{itemize}
            \item Problems encountered and limitations for actual Management nodes
            \item High Availability ? Persistent storage ? Financial aspect ?
        \end{itemize}
        
    \item Introduction to Kubernetes and its Potential Role in Managing High Availability:
    \begin{itemize}
          \item  provide an overview of Kubernetes and its potential role in managing high availability within HPC clusters. 
          \item Kubernetes offers a powerful framework for orchestrating containerized applications and can potentially enhance the availability and resilience of HPC environments.
    \end{itemize}
   \item  Possibility to Explore Slurm (Management Node) to Kubernetes and its Different Approaches:
    \begin{itemize}
    \item consider the possibility of integrating Slurm, particularly its management node, with Kubernetes. This exploration could lead to various approaches for enhancing the management and efficiency of HPC clusters, which will be a focus
        \begin{itemize}
            \item Over approach
            \item  Distant
            \item Adjacent approach 
            \item Under approach       
        \end{itemize}
    \end{itemize}

\end{itemize}

\section{Review of High Availability Solutions in HPC Cluster }
\begin{itemize}
    \item  Kubernetes-slurm, which approach ?
    \item Proposed solution to solve encountered problems for HA
    \item the use of Kubernetes as the container management framework 

    \item Presentation of the solution :
        \begin{itemize}
            \item (Kubernetes, Slurm) architecture
            \item Discuss the High Availability of management node + persistent storage volume provided by Kubernetes
        \end{itemize}
\end{itemize}

\section{Methodology}
\begin{itemize}
    \item Selection of tools and technologies required for the study
        \begin{itemize}
            \item Docker, VM, Kubernetes, Slurm, (Ansible, HelmChart)
        \end{itemize}
    \item Setting up the HPC cluster prototype with Kubernetes
        \begin{itemize}
            \item K8s cluster: Pods (Slurmctld, Slurmdbd, MySQL), Service (loadbalancer), Persistent volume
        \end{itemize}
    \item Ensure communication with K8S cluster and external resources
        \begin{itemize}
            \item Load balancer service
            \item Make the load balancer IP static (using metalLB loadbalancer) 
            \item Persistent storage (PVC + PV ---> need of local storage or Rook-Ceph 
        \end{itemize}
    \item Methods for High availability analysis
        \begin{itemize}
            \item Deployment infrastructure as code (Ansible vs HelmChart)
            \item Keep the cluster running despite issues and anomalies that may occur (having a backup)
            \item Persistence of the data required by the management services
            
            
        \end{itemize}
\end{itemize}

\section{Results Analysis}
\begin{itemize}
    \item Presentation of the prototyping results
    \item Evaluation of availability of the implemented solutions
        \begin{itemize}
            \item Test K8s cluster HA, with scenarios 
        \end{itemize}
    \item Comparison with current state

\end{itemize}

\section{Discussion}
\begin{itemize}
     \item Potential challenges and complexities encountered during the implementation and operation of the high availability solutions
    \item Limitations of the analysis and proposed solutions
    
\end{itemize}

\section{Conclusion}
\begin{itemize}
    \item Summary of the main conclusions of analysis
    \item Future prospects for research in this area
    \item What's can be improved 

\end{itemize}


\end{document}
