\documentclass{article}
\usepackage[utf8]{inputenc}

\title{Study, Prototyping, and Analysis of High Availability Solutions for an HPC Cluster Using Kubernetes}
\author{ Badr-Eddine Aji}
\date{}

\begin{document}

\maketitle

\section{Introduction}
\begin{itemize}
    \item What is HPC, what it is used for, bare-metal aspects
    \item Problem statement and objectives
    \item Research methodology, including the use of Kubernetes as the container management framework
\item Remarks ! Make sure to distinguish between the management nodes and the nodes running the HPC jobs. It could be useful to have some numbers about how many jobs are managed by an HPC cluster such as the ones managed by CISM (e.g., number of jobs per day and some characteristics, number of worker machines, etc.)
- You do not provide much detail about the problem statement: Make sure to have a convincing case for why it would make sense to manage the management nodes using a container management system rather than dedicating a few baremetal machines or VMs. Possibly, going through a few examples and scenarios can help here (e.g., handling of a machine fault, supporting a reconfiguration, scaling out the management cluster, etc.) to show why managing a fleet of containers can be more flexible.
\end{itemize}

\section{Review of High Availability Solutions in HPC Cluster }
\begin{itemize}
    \item Presentation of the typical (bare-metal) architecture of an HPC cluster
    \item Introduction to Kubernetes and its potential role in managing high availability
    \item Approaches to integrate slurm to kubernetes cluster (only the controll management part) 
    \item Presentation of the bare metal architecture: focus on management nodes
           \begin{itemize}
            \item Over approach
            \item Adjacent approach 
            \item Under approach
        \end{itemize}
    \item (Kubernetes, Slurm) architecture
\end{itemize}

\section{Methodology}
\begin{itemize}
    \item Selection of tools and technologies required for the study
        \begin{itemize}
            \item Docker, VM, Kubernetes, Slurm, (Ansible, HelmChart)
        \end{itemize}
    \item Setting up the HPC cluster prototype with Kubernetes
        \begin{itemize}
            \item K8s cluster: Pods (Slurmctld, Slurmdbd, MySQL), Service (loadbalancer), Persistent volume
        \end{itemize}
    \item Ensure communication with K8S cluster and external resources
        \begin{itemize}
            \item Load balancer service
            \item Make the load balancer IP static
            \item Persistent memory
        \end{itemize}
    \item Methods for High availability analysis
        \begin{itemize}
            \item Deployment infrastructure as code (Ansible vs HelmChart)
            \item Keep the cluster running despite issues and anomalies that may occur (having a backup)
            \item Persistence of the data required by the management services
        \end{itemize}
\end{itemize}

\section{Results Analysis}
\begin{itemize}
    \item Presentation of the prototyping results
    \item Evaluation of availability of the implemented solutions
        \begin{itemize}
            \item Test K8s cluster HA, with scenarios 
        \end{itemize}
    \item Comparison with current state
           \begin{itemize}
            \item Financial aspect
            \item Reusing old servers and implementing high availability (HA) with Kubernetes.
        \end{itemize}
\end{itemize}

\section{Discussion}
\begin{itemize}
     \item Potential challenges and complexities encountered during the implementation and operation of the high availability solutions
    \item Limitations of the analysis and proposed solutions
    
\end{itemize}

\section{Conclusion}
\begin{itemize}
    \item Summary of the main conclusions of analysis
    \item Future prospects for research in this area
    \item What's can be improved 

\end{itemize}


\end{document}
